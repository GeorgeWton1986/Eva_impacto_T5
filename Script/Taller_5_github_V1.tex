% Options for packages loaded elsewhere
\PassOptionsToPackage{unicode}{hyperref}
\PassOptionsToPackage{hyphens}{url}
%
\documentclass[
]{article}
\usepackage{amsmath,amssymb}
\usepackage{iftex}
\ifPDFTeX
  \usepackage[T1]{fontenc}
  \usepackage[utf8]{inputenc}
  \usepackage{textcomp} % provide euro and other symbols
\else % if luatex or xetex
  \usepackage{unicode-math} % this also loads fontspec
  \defaultfontfeatures{Scale=MatchLowercase}
  \defaultfontfeatures[\rmfamily]{Ligatures=TeX,Scale=1}
\fi
\usepackage{lmodern}
\ifPDFTeX\else
  % xetex/luatex font selection
\fi
% Use upquote if available, for straight quotes in verbatim environments
\IfFileExists{upquote.sty}{\usepackage{upquote}}{}
\IfFileExists{microtype.sty}{% use microtype if available
  \usepackage[]{microtype}
  \UseMicrotypeSet[protrusion]{basicmath} % disable protrusion for tt fonts
}{}
\makeatletter
\@ifundefined{KOMAClassName}{% if non-KOMA class
  \IfFileExists{parskip.sty}{%
    \usepackage{parskip}
  }{% else
    \setlength{\parindent}{0pt}
    \setlength{\parskip}{6pt plus 2pt minus 1pt}}
}{% if KOMA class
  \KOMAoptions{parskip=half}}
\makeatother
\usepackage{xcolor}
\usepackage[margin=1in]{geometry}
\usepackage{color}
\usepackage{fancyvrb}
\newcommand{\VerbBar}{|}
\newcommand{\VERB}{\Verb[commandchars=\\\{\}]}
\DefineVerbatimEnvironment{Highlighting}{Verbatim}{commandchars=\\\{\}}
% Add ',fontsize=\small' for more characters per line
\usepackage{framed}
\definecolor{shadecolor}{RGB}{248,248,248}
\newenvironment{Shaded}{\begin{snugshade}}{\end{snugshade}}
\newcommand{\AlertTok}[1]{\textcolor[rgb]{0.94,0.16,0.16}{#1}}
\newcommand{\AnnotationTok}[1]{\textcolor[rgb]{0.56,0.35,0.01}{\textbf{\textit{#1}}}}
\newcommand{\AttributeTok}[1]{\textcolor[rgb]{0.13,0.29,0.53}{#1}}
\newcommand{\BaseNTok}[1]{\textcolor[rgb]{0.00,0.00,0.81}{#1}}
\newcommand{\BuiltInTok}[1]{#1}
\newcommand{\CharTok}[1]{\textcolor[rgb]{0.31,0.60,0.02}{#1}}
\newcommand{\CommentTok}[1]{\textcolor[rgb]{0.56,0.35,0.01}{\textit{#1}}}
\newcommand{\CommentVarTok}[1]{\textcolor[rgb]{0.56,0.35,0.01}{\textbf{\textit{#1}}}}
\newcommand{\ConstantTok}[1]{\textcolor[rgb]{0.56,0.35,0.01}{#1}}
\newcommand{\ControlFlowTok}[1]{\textcolor[rgb]{0.13,0.29,0.53}{\textbf{#1}}}
\newcommand{\DataTypeTok}[1]{\textcolor[rgb]{0.13,0.29,0.53}{#1}}
\newcommand{\DecValTok}[1]{\textcolor[rgb]{0.00,0.00,0.81}{#1}}
\newcommand{\DocumentationTok}[1]{\textcolor[rgb]{0.56,0.35,0.01}{\textbf{\textit{#1}}}}
\newcommand{\ErrorTok}[1]{\textcolor[rgb]{0.64,0.00,0.00}{\textbf{#1}}}
\newcommand{\ExtensionTok}[1]{#1}
\newcommand{\FloatTok}[1]{\textcolor[rgb]{0.00,0.00,0.81}{#1}}
\newcommand{\FunctionTok}[1]{\textcolor[rgb]{0.13,0.29,0.53}{\textbf{#1}}}
\newcommand{\ImportTok}[1]{#1}
\newcommand{\InformationTok}[1]{\textcolor[rgb]{0.56,0.35,0.01}{\textbf{\textit{#1}}}}
\newcommand{\KeywordTok}[1]{\textcolor[rgb]{0.13,0.29,0.53}{\textbf{#1}}}
\newcommand{\NormalTok}[1]{#1}
\newcommand{\OperatorTok}[1]{\textcolor[rgb]{0.81,0.36,0.00}{\textbf{#1}}}
\newcommand{\OtherTok}[1]{\textcolor[rgb]{0.56,0.35,0.01}{#1}}
\newcommand{\PreprocessorTok}[1]{\textcolor[rgb]{0.56,0.35,0.01}{\textit{#1}}}
\newcommand{\RegionMarkerTok}[1]{#1}
\newcommand{\SpecialCharTok}[1]{\textcolor[rgb]{0.81,0.36,0.00}{\textbf{#1}}}
\newcommand{\SpecialStringTok}[1]{\textcolor[rgb]{0.31,0.60,0.02}{#1}}
\newcommand{\StringTok}[1]{\textcolor[rgb]{0.31,0.60,0.02}{#1}}
\newcommand{\VariableTok}[1]{\textcolor[rgb]{0.00,0.00,0.00}{#1}}
\newcommand{\VerbatimStringTok}[1]{\textcolor[rgb]{0.31,0.60,0.02}{#1}}
\newcommand{\WarningTok}[1]{\textcolor[rgb]{0.56,0.35,0.01}{\textbf{\textit{#1}}}}
\usepackage{graphicx}
\makeatletter
\def\maxwidth{\ifdim\Gin@nat@width>\linewidth\linewidth\else\Gin@nat@width\fi}
\def\maxheight{\ifdim\Gin@nat@height>\textheight\textheight\else\Gin@nat@height\fi}
\makeatother
% Scale images if necessary, so that they will not overflow the page
% margins by default, and it is still possible to overwrite the defaults
% using explicit options in \includegraphics[width, height, ...]{}
\setkeys{Gin}{width=\maxwidth,height=\maxheight,keepaspectratio}
% Set default figure placement to htbp
\makeatletter
\def\fps@figure{htbp}
\makeatother
\setlength{\emergencystretch}{3em} % prevent overfull lines
\providecommand{\tightlist}{%
  \setlength{\itemsep}{0pt}\setlength{\parskip}{0pt}}
\setcounter{secnumdepth}{-\maxdimen} % remove section numbering
\usepackage{booktabs}
\usepackage{longtable}
\usepackage{array}
\usepackage{multirow}
\usepackage{wrapfig}
\usepackage{float}
\usepackage{colortbl}
\usepackage{pdflscape}
\usepackage{tabu}
\usepackage{threeparttable}
\usepackage{threeparttablex}
\usepackage[normalem]{ulem}
\usepackage{makecell}
\usepackage{xcolor}
\usepackage{tabularray}
\usepackage[normalem]{ulem}
\usepackage{graphicx}
\UseTblrLibrary{booktabs}
\UseTblrLibrary{rotating}
\UseTblrLibrary{siunitx}
\NewTableCommand{\tinytableDefineColor}[3]{\definecolor{#1}{#2}{#3}}
\newcommand{\tinytableTabularrayUnderline}[1]{\underline{#1}}
\newcommand{\tinytableTabularrayStrikeout}[1]{\sout{#1}}
\ifLuaTeX
  \usepackage{selnolig}  % disable illegal ligatures
\fi
\usepackage{bookmark}
\IfFileExists{xurl.sty}{\usepackage{xurl}}{} % add URL line breaks if available
\urlstyle{same}
\hypersetup{
  pdftitle={EVALUACIÓN DE IMPACTO - TALLER 5},
  pdfauthor={VIÁFARA MORALES, JORGE ELIECER},
  hidelinks,
  pdfcreator={LaTeX via pandoc}}

\title{EVALUACIÓN DE IMPACTO - TALLER 5}
\author{VIÁFARA MORALES, JORGE ELIECER}
\date{2025-10-23}

\begin{document}
\maketitle

\subsection{Introducción}\label{introducciuxf3n}

En su artículo ``Islamic Rule and the Empowerment of the Poor and
Pious'', Meyersson (2014) investiga si la llegada al poder por parte del
Partido Islámico tiene algún efecto sobre el empoderamiento de las
mujeres en Turquía. Para esto, implementa la metodología de Regresión
Discontinua, explotando información de: (1) elecciones locales de
alcalde en Turquía del año 1994 y (2) mujeres con educación secundaria
completa en el año 2000. Concretamente, estima por MCO la siguiente
ecuación

\begin{align}
y_{i}=α+\beta m_i+f(x_i)+ε_i\quad \nonumber \\
\end{align}

Donde: \(y_i\) es la proporción de mujeres entre 15 y 20 años con
educación secundaria completa en el año 2000.

\(x_i\) es el margen de votos con el que ganó o perdió el candidato del
partido islámico.

\(f(⋅)\) es un polinomio de grado \(n\) de la variable \(x_i\). \(m_i\)
es una dicótoma que toma el valor de uno si \(x_i≥0\), es decir, si el
alcalde que llegó al poder en 1994 era del partido Islámico.

\(ε_i\) es el término del error. La ecuación es estimada en un
vecindario alrededor del corte, el cual, en este caso, es cero.

En este taller, ustedes realizarán algunas de las estimaciones hechas
por el autor e interpretarán los resultados. Para esto, deben usar una
submuestra aleatoria de la base de datos turquia.dta. Así, justo después
de abrir la base de datos -- i.e., la base completa, deben eliminar
aleatoriamente el 5\% de las observaciones y usar la base restante. La
semilla que deben usar para que sus resultados sean replicables es su
código de estudiante.

Como su solución a este conjunto de problemas, proporcione un documento
PDF con sus respuestas y el do file (u otro programa) que usó para
resolverlo. Se recomienda tener respuestas tan breves como sea posible.
Ciertas preguntas, sobre todo aquellas que preguntan por su opinión,
pueden no tener una única respuesta correcta. Así, lo importante es que
argumenten de manera coherente. No seguir estas instrucciones básicas
resultará en penalizaciones en su calificación.

\subsection{Primer Punto}\label{primer-punto}

\begin{enumerate}
\def\labelenumi{\arabic{enumi}.}
\tightlist
\item
  ¿Por qué el autor usa la metodología de Regresión Discontinua para
  identificar los efectos de interés? ¿Cuál es la intuición detrás?
  ¿Cuál es el supuesto de identificación?
\end{enumerate}

\subsubsection{Solución Primer Punto}\label{soluciuxf3n-primer-punto}

\begin{enumerate}
\def\labelenumi{\alph{enumi})}
\tightlist
\item
  ¿Por qué el autor usa la metodología de Regresión Discontinua para
  identificar los efectos de interés?
\end{enumerate}

\textbf{Respuesta:} El autor usa la metodología de Regresión Discontinua
(RD) porque quiere identificar el impacto causal local de las leyes o
políticas islamicas en terminos de resultados de la educación a nivel
municipal, haciando una distinción entre ganar o perder las elecciones
por parte del partido islamico.

Es decir, es conocido que un gobierno islamico representa pobreza,
religiosidad, tradiciones conservadoras y derechos restringidos para las
mujeres.

\begin{enumerate}
\def\labelenumi{\alph{enumi})}
\setcounter{enumi}{1}
\tightlist
\item
  ¿Cuál es la intuición detrás?
\end{enumerate}

\textbf{Respuesta:} El diseño de un RD permite establecer un punto de
corte (Umbral) \(c=0\) de acuerdo con el indice de elegibilidad \(m_i\).

\begin{enumerate}
\def\labelenumi{\alph{enumi})}
\setcounter{enumi}{2}
\tightlist
\item
  ¿Cuál es el supuesto de identificación?
\end{enumerate}

\textbf{Respuesta:} Las caracteristica observables y no observables
varian levemente alrededor del umbral. Además, no hay manipulación de la
variable de asignación \(x_i\) alrededor del umbral.

\subsection{Segundo Punto}\label{segundo-punto}

\begin{enumerate}
\def\labelenumi{\arabic{enumi}.}
\setcounter{enumi}{1}
\tightlist
\item
  Para cada género, presenten en una tabla los resultados de estimar las
  siguientes especificaciones
\end{enumerate}

\subsubsection{Generalidades de R}\label{generalidades-de-r}

\begin{Shaded}
\begin{Highlighting}[]
\CommentTok{\#Limpiar la consola}
\FunctionTok{cat}\NormalTok{(}\StringTok{"}\SpecialCharTok{\textbackslash{}f}\StringTok{"}\NormalTok{)}
\end{Highlighting}
\end{Shaded}

\newpage{}

\begin{Shaded}
\begin{Highlighting}[]
\CommentTok{\#Limpiar el Global Environment}
\FunctionTok{rm}\NormalTok{(}\AttributeTok{list =} \FunctionTok{ls}\NormalTok{())}

\CommentTok{\#Incluir las librerias}
\ControlFlowTok{if}\NormalTok{(}\SpecialCharTok{!}\FunctionTok{require}\NormalTok{(pacman)) }\FunctionTok{install.packages}\NormalTok{(}\StringTok{"pacman"}\NormalTok{) ; }\FunctionTok{require}\NormalTok{(pacman)}
\end{Highlighting}
\end{Shaded}

\begin{verbatim}
## Loading required package: pacman
\end{verbatim}

\begin{Shaded}
\begin{Highlighting}[]
\FunctionTok{p\_load}\NormalTok{(haven,   }\CommentTok{\#Leer archivos .dta}
\NormalTok{      dplyr,    }\CommentTok{\#Manipular datos}
\NormalTok{      stargazer,  }\CommentTok{\#Visualizar tablas de regresión}
\NormalTok{      fixest,   }\CommentTok{\#Calcular los efectos fijos}
\NormalTok{      plm,  }\CommentTok{\#Datos de panel}
\NormalTok{      knitr,   }\CommentTok{\#Visualizar tablas adicionales}
\NormalTok{      ivreg,  }\CommentTok{\#Regresiones por variables instrumentales}
\NormalTok{      broom, }\CommentTok{\#Extraer resultados de regresiones}
\NormalTok{      AER,    }\CommentTok{\#Regresiones por variables instrumentales}
\NormalTok{      sandwich, }\CommentTok{\#Errores estándar robustos}
\NormalTok{      lmtest,  }\CommentTok{\#Pruebas estadísticas}
\NormalTok{      kableExtra, }\CommentTok{\#Tablas avanzadas}
\NormalTok{      tidyverse, }\CommentTok{\#Conjunto de paquetes para ciencia de datos}
\NormalTok{      modelsummary, }\CommentTok{\#Resumir modelos}
\NormalTok{      ggplot2, }\CommentTok{\#Gráficos}
\NormalTok{      rddensity }\CommentTok{\#Test de McCrary}
\NormalTok{      )}
\end{Highlighting}
\end{Shaded}

\subsubsection{Transformación de datos en
R}\label{transformaciuxf3n-de-datos-en-r}

\begin{Shaded}
\begin{Highlighting}[]
\CommentTok{\# Definir URL del repositorio}
\NormalTok{github\_url }\OtherTok{\textless{}{-}} \StringTok{"https://raw.githubusercontent.com/GeorgeWton1986/Eva\_impacto\_T5/main"}
\CommentTok{\# Cargar datos}
\NormalTok{BD\_0 }\OtherTok{\textless{}{-}} \FunctionTok{read\_dta}\NormalTok{(}\FunctionTok{paste0}\NormalTok{(github\_url, }\StringTok{"/Data/turquia.dta"}\NormalTok{))}


\CommentTok{\# Ver estructura de los datos}
\CommentTok{\#glimpse(BD\_0)}
\CommentTok{\#head(BD\_0)}

\CommentTok{\# Verificar los nombres de la columnas}
\FunctionTok{colnames}\NormalTok{(BD\_0)}
\end{Highlighting}
\end{Shaded}

\begin{verbatim}
##  [1] "id"             "prov"           "prov_num"       "ilmvsm1994"    
##  [5] "hischshr1520f"  "i98"            "ageshr19"       "ageshr60"      
##  [9] "buyuk"          "hischshr1520m"  "i89"            "lpop1994"      
## [13] "merkezi"        "merkezp"        "partycount"     "sexr"          
## [17] "shhs"           "subbuyuk"       "vshr_islam1994"
\end{verbatim}

\begin{Shaded}
\begin{Highlighting}[]
\CommentTok{\# Incluir la semilla con mi código de estudiante}
\FunctionTok{set.seed}\NormalTok{(}\DecValTok{202415176}\NormalTok{)}

\CommentTok{\# Eliminar aleatoriamente el 5\% de las observaciones}
\NormalTok{BD\_1 }\OtherTok{\textless{}{-}}\NormalTok{ BD\_0 }\SpecialCharTok{\%\textgreater{}\%} \FunctionTok{sample\_frac}\NormalTok{(}\FloatTok{0.95}\NormalTok{)}

\CommentTok{\# Incluir la variable dicotoma, si gano el partido Islamico en 1994}

\NormalTok{BD\_1 }\OtherTok{\textless{}{-}}\NormalTok{ BD\_1 }\SpecialCharTok{\%\textgreater{}\%}
  \FunctionTok{mutate}\NormalTok{(}
    \AttributeTok{islamic\_mayor\_1994 =} \FunctionTok{ifelse}\NormalTok{(ilmvsm1994 }\SpecialCharTok{\textgreater{}=} \DecValTok{0}\NormalTok{, }\DecValTok{1}\NormalTok{, }\DecValTok{0}\NormalTok{)}
\NormalTok{  )}

\CommentTok{\# Verificar dimensiones}
\FunctionTok{cat}\NormalTok{(}\StringTok{"Observaciones después de eliminar 5\%:"}\NormalTok{, }\FunctionTok{nrow}\NormalTok{(BD\_1), }\StringTok{"}\SpecialCharTok{\textbackslash{}n}\StringTok{"}\NormalTok{)}
\end{Highlighting}
\end{Shaded}

\begin{verbatim}
## Observaciones después de eliminar 5%: 2498
\end{verbatim}

\begin{Shaded}
\begin{Highlighting}[]
\CommentTok{\# Valor de la banda optimo}

\NormalTok{h\_hat }\OtherTok{\textless{}{-}} \FloatTok{0.24}

\CommentTok{\# Submuestra 1}

\NormalTok{BD\_h }\OtherTok{\textless{}{-}}\NormalTok{ BD\_1 }\SpecialCharTok{\%\textgreater{}\%}
  \FunctionTok{filter}\NormalTok{(}\FunctionTok{abs}\NormalTok{(ilmvsm1994) }\SpecialCharTok{\textless{}=}\NormalTok{ h\_hat)}

\CommentTok{\# Submuestra 2}

\NormalTok{BD\_h2 }\OtherTok{\textless{}{-}}\NormalTok{ BD\_1 }\SpecialCharTok{\%\textgreater{}\%}
  \FunctionTok{filter}\NormalTok{(}\FunctionTok{abs}\NormalTok{(ilmvsm1994) }\SpecialCharTok{\textless{}=}\NormalTok{ h\_hat}\SpecialCharTok{/}\DecValTok{2}\NormalTok{)}

\FunctionTok{cat}\NormalTok{(}\StringTok{"Observaciones completa:"}\NormalTok{, }\FunctionTok{nrow}\NormalTok{(BD\_1), }\StringTok{"}\SpecialCharTok{\textbackslash{}n}\StringTok{"}\NormalTok{)}
\end{Highlighting}
\end{Shaded}

\begin{verbatim}
## Observaciones completa: 2498
\end{verbatim}

\begin{Shaded}
\begin{Highlighting}[]
\FunctionTok{cat}\NormalTok{(}\StringTok{"Observaciones en submuestra h=0,24:"}\NormalTok{, }\FunctionTok{nrow}\NormalTok{(BD\_h), }\StringTok{"}\SpecialCharTok{\textbackslash{}n}\StringTok{"}\NormalTok{)}
\end{Highlighting}
\end{Shaded}

\begin{verbatim}
## Observaciones en submuestra h=0,24: 970
\end{verbatim}

\begin{Shaded}
\begin{Highlighting}[]
\FunctionTok{cat}\NormalTok{(}\StringTok{"Observaciones en submuestra h/2=0,12:"}\NormalTok{, }\FunctionTok{nrow}\NormalTok{(BD\_h2), }\StringTok{"}\SpecialCharTok{\textbackslash{}n}\StringTok{"}\NormalTok{)}
\end{Highlighting}
\end{Shaded}

\begin{verbatim}
## Observaciones en submuestra h/2=0,12: 563
\end{verbatim}

\begin{Shaded}
\begin{Highlighting}[]
\CommentTok{\#Controles}

\NormalTok{controles }\OtherTok{\textless{}{-}} \FunctionTok{c}\NormalTok{(}\StringTok{"vshr\_islam1994"}\NormalTok{, }\StringTok{"partycount"}\NormalTok{, }\StringTok{"lpop1994"}\NormalTok{,}
               \StringTok{"ageshr19"}\NormalTok{, }\StringTok{"ageshr60"}\NormalTok{, }\StringTok{"sexr"}\NormalTok{, }\StringTok{"shhs"}\NormalTok{,}
               \StringTok{"merkezi"}\NormalTok{, }\StringTok{"merkezp"}\NormalTok{, }\StringTok{"buyuk"}\NormalTok{, }\StringTok{"subbuyuk"}\NormalTok{)}

\CommentTok{\#Incluir los controles a la base principal}

\NormalTok{controles\_disponibles }\OtherTok{\textless{}{-}}\NormalTok{ controles[controles }\SpecialCharTok{\%in\%} \FunctionTok{colnames}\NormalTok{(BD\_1)]}
\FunctionTok{cat}\NormalTok{(}\StringTok{"Controles disponibles:"}\NormalTok{, }\FunctionTok{paste}\NormalTok{(controles\_disponibles, }\AttributeTok{collapse=}\StringTok{", "}\NormalTok{), }\StringTok{"}\SpecialCharTok{\textbackslash{}n}\StringTok{"}\NormalTok{)}
\end{Highlighting}
\end{Shaded}

\begin{verbatim}
## Controles disponibles: vshr_islam1994, partycount, lpop1994, ageshr19, ageshr60, sexr, shhs, merkezi, merkezp, buyuk, subbuyuk
\end{verbatim}

\textbf{Respuesta:} A continuación, presentamos el desarrollo de los
modelos.

Manual avanzado de R markdown: \url{https://rpubs.com/ricardo/14631}

\subsubsection{Solución Segundo Punto}\label{soluciuxf3n-segundo-punto}

\begin{enumerate}
\def\labelenumi{\alph{enumi})}
\tightlist
\item
  Ecuación principal, para toda la muestra, sin incluir controles.
\end{enumerate}

\begin{Shaded}
\begin{Highlighting}[]
\CommentTok{\# Modelo a}

\CommentTok{\# Caso mujeres}

\NormalTok{modelo\_2m\_a }\OtherTok{\textless{}{-}} \FunctionTok{feols}\NormalTok{(}
\NormalTok{  hischshr1520f }\SpecialCharTok{\textasciitilde{}}\NormalTok{ islamic\_mayor\_1994,}
  \AttributeTok{data =}\NormalTok{ BD\_1,}
  \AttributeTok{cluster =} \SpecialCharTok{\textasciitilde{}}\NormalTok{prov}
\NormalTok{)}

\CommentTok{\# Caso hombre}

\NormalTok{modelo\_2h\_a }\OtherTok{\textless{}{-}} \FunctionTok{feols}\NormalTok{(}
\NormalTok{  hischshr1520m }\SpecialCharTok{\textasciitilde{}}\NormalTok{ islamic\_mayor\_1994,}
  \AttributeTok{data =}\NormalTok{ BD\_1,}
  \AttributeTok{cluster =} \SpecialCharTok{\textasciitilde{}}\NormalTok{prov}
\NormalTok{)}
\end{Highlighting}
\end{Shaded}

\begin{enumerate}
\def\labelenumi{\alph{enumi})}
\setcounter{enumi}{1}
\tightlist
\item
  Ecuación principal, para toda la muestra, con controles.
\end{enumerate}

\begin{Shaded}
\begin{Highlighting}[]
\CommentTok{\# Modelo b}

\CommentTok{\# formula compacta con controles mujeres}

\NormalTok{formula\_m\_b }\OtherTok{\textless{}{-}} \FunctionTok{as.formula}\NormalTok{(}\FunctionTok{paste}\NormalTok{(}
  \StringTok{"hischshr1520f \textasciitilde{} islamic\_mayor\_1994 + ilmvsm1994 + ilmvsm1994:islamic\_mayor\_1994 +"}\NormalTok{,}
  \FunctionTok{paste}\NormalTok{(controles\_disponibles, }\AttributeTok{collapse =} \StringTok{" + "}\NormalTok{)}
\NormalTok{))}

\CommentTok{\# Especificación completa con controles mujeres}

\NormalTok{modelo\_2m\_b }\OtherTok{\textless{}{-}} \FunctionTok{feols}\NormalTok{(}
\NormalTok{  formula\_m\_b,}
  \AttributeTok{data =}\NormalTok{ BD\_1,}
  \AttributeTok{cluster =} \SpecialCharTok{\textasciitilde{}}\NormalTok{prov}
\NormalTok{)}

\CommentTok{\# formula compacta con controles hombre}

\NormalTok{formula\_h\_b }\OtherTok{\textless{}{-}} \FunctionTok{as.formula}\NormalTok{(}\FunctionTok{paste}\NormalTok{(}
  \StringTok{"hischshr1520m \textasciitilde{} islamic\_mayor\_1994 + ilmvsm1994 + ilmvsm1994:islamic\_mayor\_1994 +"}\NormalTok{,}
  \FunctionTok{paste}\NormalTok{(controles\_disponibles, }\AttributeTok{collapse =} \StringTok{" + "}\NormalTok{)}
\NormalTok{))}

\CommentTok{\# Especificación completa con controles hombres}

\NormalTok{modelo\_2h\_b }\OtherTok{\textless{}{-}} \FunctionTok{feols}\NormalTok{(}
\NormalTok{  formula\_h\_b,}
  \AttributeTok{data =}\NormalTok{ BD\_1,}
  \AttributeTok{cluster =} \SpecialCharTok{\textasciitilde{}}\NormalTok{prov}
\NormalTok{)}
\end{Highlighting}
\end{Shaded}

\begin{enumerate}
\def\labelenumi{\alph{enumi})}
\setcounter{enumi}{2}
\tightlist
\item
  Ecuación principal, para la submuestra a \(\hat{h}\) unidades
  alrededor del corte, con controles.
\end{enumerate}

\begin{Shaded}
\begin{Highlighting}[]
\CommentTok{\# Modelo c}

\CommentTok{\# Especificación con submuestra $\textbackslash{}hat\{h\}$ = 0,24 mujeres}

\NormalTok{modelo\_2m\_c }\OtherTok{\textless{}{-}} \FunctionTok{feols}\NormalTok{(}
\NormalTok{  formula\_m\_b,}
  \AttributeTok{data =}\NormalTok{ BD\_h,}
  \AttributeTok{cluster =} \SpecialCharTok{\textasciitilde{}}\NormalTok{prov}
\NormalTok{)}

\CommentTok{\# Especificación con submuestra $\textbackslash{}hat\{h\}$ = 0,24 hombres}

\NormalTok{modelo\_2h\_c }\OtherTok{\textless{}{-}} \FunctionTok{feols}\NormalTok{(}
\NormalTok{  formula\_h\_b,}
  \AttributeTok{data =}\NormalTok{ BD\_h,}
  \AttributeTok{cluster =} \SpecialCharTok{\textasciitilde{}}\NormalTok{prov}
\NormalTok{)}
\end{Highlighting}
\end{Shaded}

\begin{enumerate}
\def\labelenumi{\alph{enumi})}
\setcounter{enumi}{3}
\tightlist
\item
  Ecuación principal, para la submuestra a \(\hat{h}/2\) unidades
  alrededor del corte, con controles.
\end{enumerate}

donde \(\hat{h}=0,24\) es el ancho de banda óptimo estimado por los
autores. Para las especificaciones con controles, supongan que:

\begin{align}
f(x_i)=\gamma x_i+\delta x_i×m_i \quad \nonumber \\
\end{align}

Es decir, un polinomio de grado uno con pendiente distinta a cada lado
del corte. Para las especificaciones sin controles, no incluyan ningún
polinomio. Todas las especificaciones deben usar errores estándar
clúster a nivel de provincia.

\begin{Shaded}
\begin{Highlighting}[]
\CommentTok{\# Modelo d}

\CommentTok{\# Especificación con submuestra $\textbackslash{}hat\{h\}/2$ = 0,12 mujeres}

\NormalTok{modelo\_2m\_d }\OtherTok{\textless{}{-}} \FunctionTok{feols}\NormalTok{(}
\NormalTok{  formula\_m\_b,}
  \AttributeTok{data =}\NormalTok{ BD\_h2,}
  \AttributeTok{cluster =} \SpecialCharTok{\textasciitilde{}}\NormalTok{prov}
\NormalTok{)}

\CommentTok{\# Especificación con submuestra $\textbackslash{}hat\{h\}/2$ = 0,12 hombres}

\NormalTok{modelo\_2h\_d }\OtherTok{\textless{}{-}} \FunctionTok{feols}\NormalTok{(}
\NormalTok{  formula\_h\_b,}
  \AttributeTok{data =}\NormalTok{ BD\_h2,}
  \AttributeTok{cluster =} \SpecialCharTok{\textasciitilde{}}\NormalTok{prov}
\NormalTok{)}
\end{Highlighting}
\end{Shaded}

A continuación, se presenta la tabla de resultados.

\begin{Shaded}
\begin{Highlighting}[]
\CommentTok{\# Tabla de resultados para mujeres entre 15 y 20 años educadas hasta secundaria.}

\NormalTok{modelos\_mujeres }\OtherTok{\textless{}{-}} \FunctionTok{list}\NormalTok{(}
  \StringTok{"(a) Sin controles"} \OtherTok{=}\NormalTok{ modelo\_2m\_a,}
  \StringTok{"(b) Con controles"} \OtherTok{=}\NormalTok{ modelo\_2m\_b,}
  \StringTok{"(c) h=0.24"} \OtherTok{=}\NormalTok{ modelo\_2m\_c,}
  \StringTok{"(d) h=0.12"} \OtherTok{=}\NormalTok{ modelo\_2m\_d}
\NormalTok{)}

\CommentTok{\#Table 1 Efecto en la educación de la mujeres}

\FunctionTok{modelsummary}\NormalTok{(}
\NormalTok{  modelos\_mujeres,}
  \AttributeTok{stars =} \FunctionTok{c}\NormalTok{(}\StringTok{\textquotesingle{}*\textquotesingle{}} \OtherTok{=} \FloatTok{0.1}\NormalTok{, }\StringTok{\textquotesingle{}**\textquotesingle{}} \OtherTok{=} \FloatTok{0.05}\NormalTok{, }\StringTok{\textquotesingle{}***\textquotesingle{}} \OtherTok{=} \FloatTok{0.01}\NormalTok{),}
  \AttributeTok{coef\_rename =} \FunctionTok{c}\NormalTok{(}\StringTok{"islamic\_mayor\_1994"} \OtherTok{=} \StringTok{"Islamic Mayor 1994"}\NormalTok{),}
  \AttributeTok{gof\_map =} \FunctionTok{c}\NormalTok{(}\StringTok{"nobs"}\NormalTok{, }\StringTok{"r.squared"}\NormalTok{),}
  \AttributeTok{title =} \StringTok{"Proporción mujeres 15{-}20 con secundaria completa"}
\NormalTok{)}
\end{Highlighting}
\end{Shaded}

\begin{table}
\centering
\begin{talltblr}[         %% tabularray outer open
caption={Proporción mujeres 15-20 con secundaria completa},
note{}={* p \num{< 0.1}, ** p \num{< 0.05}, *** p \num{< 0.01}},
]                     %% tabularray outer close
{                     %% tabularray inner open
colspec={Q[]Q[]Q[]Q[]Q[]},
column{1}={halign=l,},
column{2}={halign=c,},
column{3}={halign=c,},
column{4}={halign=c,},
column{5}={halign=c,},
hline{32}={1,2,3,4,5}{solid, 0.05em, black},
}                     %% tabularray inner close
\toprule
& (a) Sin controles & (b) Con controles & (c) h=0.24 & (d) h=0.12 \\ \midrule %% TinyTableHeader
(Intercept)                   & \num{0.166}***  & \num{0.572}***  & \num{0.491}***  & \num{0.456}***  \\
& (\num{0.006})   & (\num{0.039})   & (\num{0.067})   & (\num{0.079})   \\
Islamic Mayor 1994            & \num{-0.027}*** & \num{0.013}*    & \num{0.025}***  & \num{0.030}**   \\
& (\num{0.006})   & (\num{0.007})   & (\num{0.007})   & (\num{0.011})   \\
ilmvsm1994                    &                  & \num{0.064}***  & \num{0.018}     & \num{-0.124}    \\
&                  & (\num{0.017})   & (\num{0.045})   & (\num{0.101})   \\
vshr\_islam1994              &                  & \num{-0.002}*** & \num{-0.001}*** & \num{-0.002}*** \\
&                  & (\num{0.000})   & (\num{0.000})   & (\num{0.000})   \\
partycount                    &                  & \num{-0.008}*** & \num{-0.007}*** & \num{-0.008}*** \\
&                  & (\num{0.002})   & (\num{0.002})   & (\num{0.003})   \\
lpop1994                      &                  & \num{0.007}**   & \num{0.008}**   & \num{0.008}*    \\
&                  & (\num{0.003})   & (\num{0.003})   & (\num{0.005})   \\
ageshr19                      &                  & \num{-0.008}*** & \num{-0.008}*** & \num{-0.008}*** \\
&                  & (\num{0.001})   & (\num{0.001})   & (\num{0.001})   \\
ageshr60                      &                  & \num{-0.006}*** & \num{-0.005}*** & \num{-0.004}*   \\
&                  & (\num{0.001})   & (\num{0.002})   & (\num{0.002})   \\
sexr                          &                  & \num{0.000}***  & \num{0.000}     & \num{0.000}     \\
&                  & (\num{0.000})   & (\num{0.000})   & (\num{0.000})   \\
shhs                          &                  & \num{0.005}**   & \num{0.004}***  & \num{0.005}***  \\
&                  & (\num{0.002})   & (\num{0.001})   & (\num{0.002})   \\
merkezi                       &                  & \num{0.067}***  & \num{0.060}***  & \num{0.061}***  \\
&                  & (\num{0.006})   & (\num{0.007})   & (\num{0.008})   \\
merkezp                       &                  & \num{0.056}***  & \num{0.063}***  & \num{0.057}***  \\
&                  & (\num{0.010})   & (\num{0.012})   & (\num{0.016})   \\
buyuk                         &                  & \num{0.062}***  & \num{0.070}***  & \num{0.069}***  \\
&                  & (\num{0.016})   & (\num{0.017})   & (\num{0.025})   \\
subbuyuk                      &                  & \num{0.044}***  & \num{0.045}**   & \num{0.038}**   \\
&                  & (\num{0.014})   & (\num{0.017})   & (\num{0.016})   \\
Islamic Mayor 1994:ilmvsm1994 &                  & \num{-0.045}    & \num{-0.123}*   & \num{0.098}     \\
&                  & (\num{0.043})   & (\num{0.068})   & (\num{0.150})   \\
Num.Obs.                      & \num{2498}      & \num{2498}      & \num{970}       & \num{563}       \\
R2                            & \num{0.009}     & \num{0.447}     & \num{0.557}     & \num{0.527}     \\
\bottomrule
\end{talltblr}
\end{table}

\begin{Shaded}
\begin{Highlighting}[]
\CommentTok{\# Tabla de resultados para hombres entre 15 y 20 años educadas hasta secundaria.}

\NormalTok{modelos\_hombres }\OtherTok{\textless{}{-}} \FunctionTok{list}\NormalTok{(}
  \StringTok{"(a) Sin controles"} \OtherTok{=}\NormalTok{ modelo\_2h\_a,}
  \StringTok{"(b) Con controles"} \OtherTok{=}\NormalTok{ modelo\_2h\_b,}
  \StringTok{"(c) h=0.24"} \OtherTok{=}\NormalTok{ modelo\_2h\_c,}
  \StringTok{"(d) h=0.12"} \OtherTok{=}\NormalTok{ modelo\_2h\_d}
\NormalTok{)}
\end{Highlighting}
\end{Shaded}

Resultado del efecto en la eduacion de las mujeres:

\begin{Shaded}
\begin{Highlighting}[]
\CommentTok{\#Table 1 Efecto en la educación de la mujeres}

\FunctionTok{cat}\NormalTok{(}\StringTok{"}\SpecialCharTok{\textbackslash{}n}\StringTok{=== TABLA 1: EFECTOS EN EDUCACIÓN {-} MUJERES ===}\SpecialCharTok{\textbackslash{}n}\StringTok{"}\NormalTok{)}
\end{Highlighting}
\end{Shaded}

\begin{verbatim}
## 
## === TABLA 1: EFECTOS EN EDUCACIÓN - MUJERES ===
\end{verbatim}

\begin{Shaded}
\begin{Highlighting}[]
\FunctionTok{modelsummary}\NormalTok{(}
\NormalTok{  modelos\_mujeres,}
  \AttributeTok{stars =} \FunctionTok{c}\NormalTok{(}\StringTok{\textquotesingle{}*\textquotesingle{}} \OtherTok{=} \FloatTok{0.1}\NormalTok{, }\StringTok{\textquotesingle{}**\textquotesingle{}} \OtherTok{=} \FloatTok{0.05}\NormalTok{, }\StringTok{\textquotesingle{}***\textquotesingle{}} \OtherTok{=} \FloatTok{0.01}\NormalTok{),}
  \AttributeTok{coef\_rename =} \FunctionTok{c}\NormalTok{(}\StringTok{"islamic\_mayor\_1994"} \OtherTok{=} \StringTok{"Islamic Mayor 1994"}\NormalTok{),}
  \AttributeTok{gof\_map =} \FunctionTok{c}\NormalTok{(}\StringTok{"nobs"}\NormalTok{, }\StringTok{"r.squared"}\NormalTok{),}
  \AttributeTok{title =} \StringTok{"Proporción mujeres 15{-}20 con secundaria completa"}
\NormalTok{)}
\end{Highlighting}
\end{Shaded}

\begin{table}
\centering
\begin{talltblr}[         %% tabularray outer open
caption={Proporción mujeres 15-20 con secundaria completa},
note{}={* p \num{< 0.1}, ** p \num{< 0.05}, *** p \num{< 0.01}},
]                     %% tabularray outer close
{                     %% tabularray inner open
colspec={Q[]Q[]Q[]Q[]Q[]},
column{1}={halign=l,},
column{2}={halign=c,},
column{3}={halign=c,},
column{4}={halign=c,},
column{5}={halign=c,},
hline{32}={1,2,3,4,5}{solid, 0.05em, black},
}                     %% tabularray inner close
\toprule
& (a) Sin controles & (b) Con controles & (c) h=0.24 & (d) h=0.12 \\ \midrule %% TinyTableHeader
(Intercept)                   & \num{0.166}***  & \num{0.572}***  & \num{0.491}***  & \num{0.456}***  \\
& (\num{0.006})   & (\num{0.039})   & (\num{0.067})   & (\num{0.079})   \\
Islamic Mayor 1994            & \num{-0.027}*** & \num{0.013}*    & \num{0.025}***  & \num{0.030}**   \\
& (\num{0.006})   & (\num{0.007})   & (\num{0.007})   & (\num{0.011})   \\
ilmvsm1994                    &                  & \num{0.064}***  & \num{0.018}     & \num{-0.124}    \\
&                  & (\num{0.017})   & (\num{0.045})   & (\num{0.101})   \\
vshr\_islam1994              &                  & \num{-0.002}*** & \num{-0.001}*** & \num{-0.002}*** \\
&                  & (\num{0.000})   & (\num{0.000})   & (\num{0.000})   \\
partycount                    &                  & \num{-0.008}*** & \num{-0.007}*** & \num{-0.008}*** \\
&                  & (\num{0.002})   & (\num{0.002})   & (\num{0.003})   \\
lpop1994                      &                  & \num{0.007}**   & \num{0.008}**   & \num{0.008}*    \\
&                  & (\num{0.003})   & (\num{0.003})   & (\num{0.005})   \\
ageshr19                      &                  & \num{-0.008}*** & \num{-0.008}*** & \num{-0.008}*** \\
&                  & (\num{0.001})   & (\num{0.001})   & (\num{0.001})   \\
ageshr60                      &                  & \num{-0.006}*** & \num{-0.005}*** & \num{-0.004}*   \\
&                  & (\num{0.001})   & (\num{0.002})   & (\num{0.002})   \\
sexr                          &                  & \num{0.000}***  & \num{0.000}     & \num{0.000}     \\
&                  & (\num{0.000})   & (\num{0.000})   & (\num{0.000})   \\
shhs                          &                  & \num{0.005}**   & \num{0.004}***  & \num{0.005}***  \\
&                  & (\num{0.002})   & (\num{0.001})   & (\num{0.002})   \\
merkezi                       &                  & \num{0.067}***  & \num{0.060}***  & \num{0.061}***  \\
&                  & (\num{0.006})   & (\num{0.007})   & (\num{0.008})   \\
merkezp                       &                  & \num{0.056}***  & \num{0.063}***  & \num{0.057}***  \\
&                  & (\num{0.010})   & (\num{0.012})   & (\num{0.016})   \\
buyuk                         &                  & \num{0.062}***  & \num{0.070}***  & \num{0.069}***  \\
&                  & (\num{0.016})   & (\num{0.017})   & (\num{0.025})   \\
subbuyuk                      &                  & \num{0.044}***  & \num{0.045}**   & \num{0.038}**   \\
&                  & (\num{0.014})   & (\num{0.017})   & (\num{0.016})   \\
Islamic Mayor 1994:ilmvsm1994 &                  & \num{-0.045}    & \num{-0.123}*   & \num{0.098}     \\
&                  & (\num{0.043})   & (\num{0.068})   & (\num{0.150})   \\
Num.Obs.                      & \num{2498}      & \num{2498}      & \num{970}       & \num{563}       \\
R2                            & \num{0.009}     & \num{0.447}     & \num{0.557}     & \num{0.527}     \\
\bottomrule
\end{talltblr}
\end{table}

\begin{Shaded}
\begin{Highlighting}[]
\CommentTok{\# Extraer coeficientes}

\NormalTok{coefs\_mujeres }\OtherTok{\textless{}{-}} \FunctionTok{sapply}\NormalTok{(modelos\_mujeres, }\ControlFlowTok{function}\NormalTok{(m) \{}
\NormalTok{  coef\_val }\OtherTok{\textless{}{-}} \FunctionTok{coef}\NormalTok{(m)[}\StringTok{"islamic\_mayor\_1994"}\NormalTok{]}
\NormalTok{  se\_val }\OtherTok{\textless{}{-}} \FunctionTok{sqrt}\NormalTok{(}\FunctionTok{vcov}\NormalTok{(m)[}\StringTok{"islamic\_mayor\_1994"}\NormalTok{, }\StringTok{"islamic\_mayor\_1994"}\NormalTok{])}
  \FunctionTok{c}\NormalTok{(}\AttributeTok{coef =}\NormalTok{ coef\_val, }\AttributeTok{se =}\NormalTok{ se\_val)}
\NormalTok{\})}

\FunctionTok{cat}\NormalTok{(}\StringTok{"}\SpecialCharTok{\textbackslash{}n}\StringTok{=== RESUMEN COEFICIENTES MUJERES ===}\SpecialCharTok{\textbackslash{}n}\StringTok{"}\NormalTok{)}
\end{Highlighting}
\end{Shaded}

\begin{verbatim}
## 
## === RESUMEN COEFICIENTES MUJERES ===
\end{verbatim}

\begin{Shaded}
\begin{Highlighting}[]
\FunctionTok{print}\NormalTok{(}\FunctionTok{t}\NormalTok{(coefs\_mujeres))}
\end{Highlighting}
\end{Shaded}

\begin{verbatim}
##                   coef.islamic_mayor_1994          se
## (a) Sin controles             -0.02730349 0.006022508
## (b) Con controles              0.01288888 0.007300374
## (c) h=0.24                     0.02513731 0.007463626
## (d) h=0.12                     0.02975422 0.011316746
\end{verbatim}

Resultado del efecto en la eduacion de los hombres:

\begin{Shaded}
\begin{Highlighting}[]
\CommentTok{\#Table 1 Efecto en la educación de la hombres}

\FunctionTok{cat}\NormalTok{(}\StringTok{"}\SpecialCharTok{\textbackslash{}n}\StringTok{=== TABLA 2: EFECTOS EN EDUCACIÓN {-} HOMBRES ===}\SpecialCharTok{\textbackslash{}n}\StringTok{"}\NormalTok{)}
\end{Highlighting}
\end{Shaded}

\begin{verbatim}
## 
## === TABLA 2: EFECTOS EN EDUCACIÓN - HOMBRES ===
\end{verbatim}

\begin{Shaded}
\begin{Highlighting}[]
\FunctionTok{modelsummary}\NormalTok{(}
\NormalTok{  modelos\_hombres,}
  \AttributeTok{stars =} \FunctionTok{c}\NormalTok{(}\StringTok{\textquotesingle{}*\textquotesingle{}} \OtherTok{=} \FloatTok{0.1}\NormalTok{, }\StringTok{\textquotesingle{}**\textquotesingle{}} \OtherTok{=} \FloatTok{0.05}\NormalTok{, }\StringTok{\textquotesingle{}***\textquotesingle{}} \OtherTok{=} \FloatTok{0.01}\NormalTok{),}
  \AttributeTok{coef\_rename =} \FunctionTok{c}\NormalTok{(}\StringTok{"islamic\_mayor\_1994"} \OtherTok{=} \StringTok{"Islamic Mayor 1994"}\NormalTok{),}
  \AttributeTok{gof\_map =} \FunctionTok{c}\NormalTok{(}\StringTok{"nobs"}\NormalTok{, }\StringTok{"r.squared"}\NormalTok{),}
  \AttributeTok{title =} \StringTok{"Proporción hombres 15{-}20 con secundaria completa"}
\NormalTok{)}
\end{Highlighting}
\end{Shaded}

\begin{table}
\centering
\begin{talltblr}[         %% tabularray outer open
caption={Proporción hombres 15-20 con secundaria completa},
note{}={* p \num{< 0.1}, ** p \num{< 0.05}, *** p \num{< 0.01}},
]                     %% tabularray outer close
{                     %% tabularray inner open
colspec={Q[]Q[]Q[]Q[]Q[]},
column{1}={halign=l,},
column{2}={halign=c,},
column{3}={halign=c,},
column{4}={halign=c,},
column{5}={halign=c,},
hline{32}={1,2,3,4,5}{solid, 0.05em, black},
}                     %% tabularray inner close
\toprule
& (a) Sin controles & (b) Con controles & (c) h=0.24 & (d) h=0.12 \\ \midrule %% TinyTableHeader
(Intercept)                   & \num{19.185}*** & \num{51.514}*** & \num{50.821}*** & \num{43.546}*** \\
& (\num{0.483})   & (\num{3.958})   & (\num{6.171})   & (\num{7.688})   \\
Islamic Mayor 1994            & \num{0.356}     & \num{0.927}     & \num{0.943}     & \num{2.151}*    \\
& (\num{0.553})   & (\num{0.693})   & (\num{0.783})   & (\num{1.108})   \\
ilmvsm1994                    &                  & \num{8.312}***  & \num{7.743}     & \num{3.223}     \\
&                  & (\num{1.403})   & (\num{4.795})   & (\num{9.155})   \\
vshr\_islam1994              &                  & \num{-0.113}*** & \num{-0.096}**  & \num{-0.124}*** \\
&                  & (\num{0.025})   & (\num{0.040})   & (\num{0.047})   \\
partycount                    &                  & \num{-1.380}*** & \num{-1.243}*** & \num{-1.244}*** \\
&                  & (\num{0.180})   & (\num{0.195})   & (\num{0.293})   \\
lpop1994                      &                  & \num{0.522}     & \num{0.431}     & \num{0.781}*    \\
&                  & (\num{0.317})   & (\num{0.337})   & (\num{0.448})   \\
ageshr19                      &                  & \num{-0.502}*** & \num{-0.545}*** & \num{-0.499}*** \\
&                  & (\num{0.061})   & (\num{0.045})   & (\num{0.059})   \\
ageshr60                      &                  & \num{-0.547}*** & \num{-0.512}*** & \num{-0.376}**  \\
&                  & (\num{0.088})   & (\num{0.118})   & (\num{0.153})   \\
sexr                          &                  & \num{-0.044}*** & \num{-0.026}    & \num{-0.009}    \\
&                  & (\num{0.010})   & (\num{0.029})   & (\num{0.036})   \\
shhs                          &                  & \num{0.547}***  & \num{0.499}***  & \num{0.530}***  \\
&                  & (\num{0.168})   & (\num{0.134})   & (\num{0.172})   \\
merkezi                       &                  & \num{4.917}***  & \num{5.428}***  & \num{5.492}***  \\
&                  & (\num{0.488})   & (\num{0.595})   & (\num{0.883})   \\
merkezp                       &                  & \num{5.784}***  & \num{5.575}***  & \num{4.162}***  \\
&                  & (\num{0.888})   & (\num{1.073})   & (\num{1.534})   \\
buyuk                         &                  & \num{10.164}*** & \num{9.477}***  & \num{6.595}***  \\
&                  & (\num{1.913})   & (\num{2.143})   & (\num{2.067})   \\
subbuyuk                      &                  & \num{5.403}***  & \num{4.725}***  & \num{2.254}*    \\
&                  & (\num{1.168})   & (\num{1.528})   & (\num{1.272})   \\
Islamic Mayor 1994:ilmvsm1994 &                  & \num{-6.125}**  & \num{-7.357}    & \num{-17.397}   \\
&                  & (\num{2.871})   & (\num{7.234})   & (\num{15.733})  \\
Num.Obs.                      & \num{2498}      & \num{2498}      & \num{970}       & \num{563}       \\
R2                            & \num{0.000}     & \num{0.208}     & \num{0.281}     & \num{0.285}     \\
\bottomrule
\end{talltblr}
\end{table}

\begin{Shaded}
\begin{Highlighting}[]
\CommentTok{\# Extraer coeficientes hombres}

\NormalTok{coefs\_hombres }\OtherTok{\textless{}{-}} \FunctionTok{sapply}\NormalTok{(modelos\_hombres, }\ControlFlowTok{function}\NormalTok{(m) \{}
\NormalTok{  coef\_val }\OtherTok{\textless{}{-}} \FunctionTok{coef}\NormalTok{(m)[}\StringTok{"islamic\_mayor\_1994"}\NormalTok{]}
\NormalTok{  se\_val }\OtherTok{\textless{}{-}} \FunctionTok{sqrt}\NormalTok{(}\FunctionTok{vcov}\NormalTok{(m)[}\StringTok{"islamic\_mayor\_1994"}\NormalTok{, }\StringTok{"islamic\_mayor\_1994"}\NormalTok{])}
  \FunctionTok{c}\NormalTok{(}\AttributeTok{coef =}\NormalTok{ coef\_val, }\AttributeTok{se =}\NormalTok{ se\_val)}
\NormalTok{\})}

\FunctionTok{cat}\NormalTok{(}\StringTok{"}\SpecialCharTok{\textbackslash{}n}\StringTok{=== RESUMEN COEFICIENTES HOMBRES ===}\SpecialCharTok{\textbackslash{}n}\StringTok{"}\NormalTok{)}
\end{Highlighting}
\end{Shaded}

\begin{verbatim}
## 
## === RESUMEN COEFICIENTES HOMBRES ===
\end{verbatim}

\begin{Shaded}
\begin{Highlighting}[]
\FunctionTok{print}\NormalTok{(}\FunctionTok{t}\NormalTok{(coefs\_hombres))}
\end{Highlighting}
\end{Shaded}

\begin{verbatim}
##                   coef.islamic_mayor_1994        se
## (a) Sin controles               0.3558749 0.5529102
## (b) Con controles               0.9268883 0.6926490
## (c) h=0.24                      0.9433412 0.7825749
## (d) h=0.12                      2.1510102 1.1080086
\end{verbatim}

\subsection{Tercer Punto}\label{tercer-punto}

\begin{enumerate}
\def\labelenumi{\arabic{enumi}.}
\setcounter{enumi}{2}
\tightlist
\item
  A partir de los resultados encontrados en el anterior punto,
  respondan:
\end{enumerate}

\subsubsection{Solución tercer Punto}\label{soluciuxf3n-tercer-punto}

\begin{enumerate}
\def\labelenumi{\alph{enumi})}
\tightlist
\item
  ¿Por qué cambian los coeficientes entre especificaciones?
\end{enumerate}

\textbf{Respuesta:} Los cambios en los coeficinetes obedecen a la
inclusión o no de controles, al tamaño de la muestra (amplitud de la
banda) y cada uno de los modelos tienen diferentes problemas de
identificación. Veamos:

\textbf{Modelo a Sin controles:} El coeficiente estimado (-0.027)***
presenta una correlación entre la variable eduación secundarias en niñas
entre los 15 y 20 años y la varible dicotoma que respesenta la victoria
del partido islamico en las elecciones de 1994. Por lo tanto, el
coeficiente puede estar sesgado, por omisión de variables y no puede ser
interpretados como un efecto causal.

\textbf{Modelo b Con controles:} El coeficiente estimado (0.012)* se
ajusta al incluir las varibales de control, a tal punto que el signo del
coeficiente cambio respecto del modelo a. Además, el ajuste de la
especificación permitio que los sesgos de selección se redujeran por lo
controles. No obsetante, los municipios son dispares entre sí.

\textbf{Modelo c h=0.24:} El coeficiente estimado (0.025)*** tiene una
especificación con controles y una amplitud de banda de 0.24, inidicado
que se realiza una comparación entre los municipios más cercanos al
umbral (\(c=0\)), asegurando asi que la selección es casi aleatoria. Por
lo tanto, se puede establecer que se minimiza el sesgo y este
coeficiente puede ser interpretado como un efecto causal local (LATE).

\textbf{Modelo d h=0.12:} El coeficiente estimado (0.029)** fue
calculado con una especificación con controles y una amplitud de banda
de 0.12, estas caracteristas disminuyen en número de observaciones y
generan homogeniedad en la muestra de municipios que se estan comparando
al rededor del umbral (\(c=0\)). Sin embargo, la disminución en el
número de observaciones puede aumentar la varianza del estimador. Ahora,
en terminos generales, el coeficiente tiene un valor cercano al modelo
c, que puede sugerir una validación por robustez.

\begin{enumerate}
\def\labelenumi{\alph{enumi})}
\setcounter{enumi}{1}
\tightlist
\item
  ¿Cuál parece ser el impacto de la llegada al poder del Partido
  Islámico para las mujeres?
\end{enumerate}

\textbf{Respuesta:} De conformidad con la tabla 1 columna 3 (0.025)***,
el efecto de la llegada al poder del Partido Islámico para las mujeres
entre 15 y 20 años con secundaria completa, es positivo y significativo
al 1\%. Entonces, su interpretación sería la siguiente: La llegada de un
alcalde islámico en 1994, incremento en promedio 2.5 puntos percetuales
la proporción de mujeres entre 15 y 20 años con educación secundaria
completa en el año 2000, en los municipios cercanos al umbral de
victoria electoral. En terminos relativos, se puede establecer que fue
de 2.5/16.3=15.34\% (Proporción promedio de mujeres con secundaria
completa en 2000).

¿Parece ser este impacto robusto a las especificaciones?

\textbf{Respuesta:} Si, el impacto es robusto cuando se comparan las
columnas 3 y 4 de la tabla 1, entendiendo que los coeficientes son
positivos y sus significancias estadisticas varian entre el 1\% y el
5\%. Puede existir otra manera de comparación frente a la columna 1,
donde la especificación genera una correlación y el estimados es
negativo. En conclusión, el cambio del signo entre los modelos confirma
el efecto causal robusto.

\subsection{Cuarto Punto}\label{cuarto-punto}

\begin{enumerate}
\def\labelenumi{\arabic{enumi}.}
\setcounter{enumi}{3}
\tightlist
\item
  Finalmente, presenten evidencia a favor (o en contra) del supuesto de
  identificación. Para esto,
\end{enumerate}

\subsubsection{Solución cuarto Punto}\label{soluciuxf3n-cuarto-punto}

\begin{enumerate}
\def\labelenumi{\alph{enumi})}
\tightlist
\item
  Presenten en una gráfica la distribución kernel o el histograma de la
  variable de asignación \(x_i\). ¿Parece haber manipulación?
\end{enumerate}

\begin{Shaded}
\begin{Highlighting}[]
\CommentTok{\# Figura 1 Histograma con 99 bins}
\NormalTok{p1 }\OtherTok{\textless{}{-}} \FunctionTok{ggplot}\NormalTok{(BD\_1, }\FunctionTok{aes}\NormalTok{(}\AttributeTok{x =}\NormalTok{ ilmvsm1994)) }\SpecialCharTok{+}
  \FunctionTok{geom\_histogram}\NormalTok{(}\AttributeTok{bins =} \DecValTok{99}\NormalTok{, }\AttributeTok{fill =} \StringTok{"steelblue"}\NormalTok{, }\AttributeTok{color =} \StringTok{"black"}\NormalTok{, }\AttributeTok{alpha =} \FloatTok{0.7}\NormalTok{) }\SpecialCharTok{+}
  \FunctionTok{geom\_vline}\NormalTok{(}\AttributeTok{xintercept =} \DecValTok{0}\NormalTok{, }\AttributeTok{linetype =} \StringTok{"dashed"}\NormalTok{, }\AttributeTok{color =} \StringTok{"red"}\NormalTok{, }\AttributeTok{size =} \FloatTok{1.2}\NormalTok{) }\SpecialCharTok{+}
  \FunctionTok{labs}\NormalTok{(}
    \AttributeTok{title =} \StringTok{"Figura 1: Test de Manipulación {-} Distribución del Margen de Victoria"}\NormalTok{,}
    \AttributeTok{subtitle =} \StringTok{"McCrary (2008) Density Test"}\NormalTok{,}
    \AttributeTok{x =} \StringTok{"Margen de victoria del partido islámico (ilmvsm1994)"}\NormalTok{,}
    \AttributeTok{y =} \StringTok{"Frecuencia"}\NormalTok{,}
    \AttributeTok{caption =} \StringTok{"Nota: Si hay salto en x=0 → evidencia de manipulación"}
\NormalTok{  ) }\SpecialCharTok{+}
  \FunctionTok{theme\_minimal}\NormalTok{() }\SpecialCharTok{+}
  \FunctionTok{theme}\NormalTok{(}
    \AttributeTok{plot.title =} \FunctionTok{element\_text}\NormalTok{(}\AttributeTok{hjust =} \FloatTok{0.5}\NormalTok{, }\AttributeTok{face =} \StringTok{"bold"}\NormalTok{),}
    \AttributeTok{plot.subtitle =} \FunctionTok{element\_text}\NormalTok{(}\AttributeTok{hjust =} \FloatTok{0.5}\NormalTok{)}
\NormalTok{  )}

\FunctionTok{print}\NormalTok{(p1)}
\end{Highlighting}
\end{Shaded}

\includegraphics{Taller_5_github_V1_files/figure-latex/unnamed-chunk-11-1.pdf}

\begin{Shaded}
\begin{Highlighting}[]
\FunctionTok{ggsave}\NormalTok{(}\StringTok{"figura1\_manipulacion.png"}\NormalTok{, }\AttributeTok{width =} \DecValTok{10}\NormalTok{, }\AttributeTok{height =} \DecValTok{6}\NormalTok{)}

\CommentTok{\# Figura 2 Densidad kernel}
\NormalTok{p2 }\OtherTok{\textless{}{-}} \FunctionTok{ggplot}\NormalTok{(BD\_1, }\FunctionTok{aes}\NormalTok{(}\AttributeTok{x =}\NormalTok{ ilmvsm1994)) }\SpecialCharTok{+}
  \FunctionTok{geom\_density}\NormalTok{(}\AttributeTok{fill =} \StringTok{"steelblue"}\NormalTok{, }\AttributeTok{alpha =} \FloatTok{0.5}\NormalTok{, }\AttributeTok{color =} \StringTok{"darkblue"}\NormalTok{, }\AttributeTok{size =} \DecValTok{1}\NormalTok{) }\SpecialCharTok{+}
  \FunctionTok{geom\_vline}\NormalTok{(}\AttributeTok{xintercept =} \DecValTok{0}\NormalTok{, }\AttributeTok{linetype =} \StringTok{"dashed"}\NormalTok{, }\AttributeTok{color =} \StringTok{"red"}\NormalTok{, }\AttributeTok{size =} \FloatTok{1.2}\NormalTok{) }\SpecialCharTok{+}
  \FunctionTok{labs}\NormalTok{(}
    \AttributeTok{title =} \StringTok{"Figura 2: Densidad Kernel del Margen de Victoria"}\NormalTok{,}
    \AttributeTok{x =} \StringTok{"Margen de victoria del partido islámico"}\NormalTok{,}
    \AttributeTok{y =} \StringTok{"Densidad"}\NormalTok{,}
    \AttributeTok{caption =} \StringTok{"Nota: Si hay salto en x=0 → evidencia de manipulación"}
\NormalTok{  ) }\SpecialCharTok{+}
  \FunctionTok{theme\_minimal}\NormalTok{() }\SpecialCharTok{+} 
    \FunctionTok{theme}\NormalTok{(}
    \AttributeTok{plot.title =} \FunctionTok{element\_text}\NormalTok{(}\AttributeTok{hjust =} \FloatTok{0.5}\NormalTok{, }\AttributeTok{face =} \StringTok{"bold"}\NormalTok{),}
    \AttributeTok{plot.subtitle =} \FunctionTok{element\_text}\NormalTok{(}\AttributeTok{hjust =} \FloatTok{0.5}\NormalTok{)}
\NormalTok{  )}

\FunctionTok{print}\NormalTok{(p2)}
\end{Highlighting}
\end{Shaded}

\includegraphics{Taller_5_github_V1_files/figure-latex/unnamed-chunk-11-2.pdf}

\begin{Shaded}
\begin{Highlighting}[]
\FunctionTok{ggsave}\NormalTok{(}\StringTok{"figura2\_densidad.png"}\NormalTok{, }\AttributeTok{width =} \DecValTok{10}\NormalTok{, }\AttributeTok{height =} \DecValTok{6}\NormalTok{)}
\end{Highlighting}
\end{Shaded}

\begin{Shaded}
\begin{Highlighting}[]
\CommentTok{\# Test formal}
\NormalTok{density\_test }\OtherTok{\textless{}{-}} \FunctionTok{rddensity}\NormalTok{(}
  \AttributeTok{X =}\NormalTok{ BD\_1}\SpecialCharTok{$}\NormalTok{ilmvsm1994,}
  \AttributeTok{c =} \DecValTok{0}\NormalTok{,}
  \AttributeTok{p =} \DecValTok{2}\NormalTok{,              }\CommentTok{\# Polinomio local orden 2}
  \AttributeTok{q =} \DecValTok{3}\NormalTok{,              }\CommentTok{\# Bias correction orden 3}
  \AttributeTok{kernel =} \StringTok{"triangular"}\NormalTok{,}
  \AttributeTok{fitselect =} \StringTok{"unrestricted"}
\NormalTok{)}

\CommentTok{\# Mostrar resumen}
\FunctionTok{summary}\NormalTok{(density\_test)}
\end{Highlighting}
\end{Shaded}

\begin{verbatim}
## 
## Manipulation testing using local polynomial density estimation.
## 
## Number of obs =       2498
## Model =               unrestricted
## Kernel =              triangular
## BW method =           estimated
## VCE method =          jackknife
## 
## c = 0                 Left of c           Right of c          
## Number of obs         2199                299                 
## Eff. Number of obs    866                 286                 
## Order est. (p)        2                   2                   
## Order bias (q)        3                   3                   
## BW est. (h)           0.293               0.285               
## 
## Method                T                   P > |T|             
## Robust                -1.2443             0.2134              
## 
## 
## P-values of binomial tests (H0: p=0.5).
## 
## Window Length / 2          <c     >=c    P>|T|
## 0.009                      20      24    0.6516
## 0.017                      40      47    0.5203
## 0.026                      68      61    0.5975
## 0.035                      91      77    0.3159
## 0.044                     125      93    0.0355
## 0.052                     148     105    0.0082
## 0.061                     176     123    0.0026
## 0.070                     201     139    0.0009
## 0.079                     220     151    0.0004
## 0.087                     246     164    0.0001
\end{verbatim}

\begin{Shaded}
\begin{Highlighting}[]
\CommentTok{\# Extraer resultados clave}
\NormalTok{test\_stat }\OtherTok{\textless{}{-}}\NormalTok{ density\_test}\SpecialCharTok{$}\NormalTok{test}\SpecialCharTok{$}\NormalTok{t\_jk}
\NormalTok{pvalue }\OtherTok{\textless{}{-}}\NormalTok{ density\_test}\SpecialCharTok{$}\NormalTok{test}\SpecialCharTok{$}\NormalTok{p\_jk}
\NormalTok{bandwidth\_izq }\OtherTok{\textless{}{-}}\NormalTok{ density\_test}\SpecialCharTok{$}\NormalTok{h}\SpecialCharTok{$}\NormalTok{left}
\NormalTok{bandwidth\_der }\OtherTok{\textless{}{-}}\NormalTok{ density\_test}\SpecialCharTok{$}\NormalTok{h}\SpecialCharTok{$}\NormalTok{right}

\CommentTok{\# Crear el objeto con histograma para obtener los datos}
\NormalTok{rdplot\_obj }\OtherTok{\textless{}{-}} \FunctionTok{rdplotdensity}\NormalTok{(}
  \AttributeTok{rdd =}\NormalTok{ density\_test,}
  \AttributeTok{X =}\NormalTok{ BD\_1}\SpecialCharTok{$}\NormalTok{ilmvsm1994,}
  \AttributeTok{plotRange =} \FunctionTok{c}\NormalTok{(}\SpecialCharTok{{-}}\FloatTok{0.5}\NormalTok{, }\FloatTok{0.5}\NormalTok{),}
  \AttributeTok{plotN =} \DecValTok{25}\NormalTok{,}
  \AttributeTok{plotGrid =} \FunctionTok{c}\NormalTok{(}\StringTok{"es"}\NormalTok{,}\StringTok{"qs"}\NormalTok{),}
  \AttributeTok{type =} \StringTok{"points"}\NormalTok{,}
  \AttributeTok{CIshade =} \FloatTok{0.2}\NormalTok{,}
  \AttributeTok{hist =} \ConstantTok{FALSE}\NormalTok{,}
  \AttributeTok{histBreaks =} \ConstantTok{NULL}\NormalTok{,}
  \AttributeTok{histFillCol =} \StringTok{"gray90"}\NormalTok{,}
  \AttributeTok{histLineCol =} \StringTok{"white"}\NormalTok{,}
  \AttributeTok{title =} \StringTok{"Figura 3: Test de McCrary"}\NormalTok{,}
  \AttributeTok{xlabel =} \StringTok{"Islamic win margin (ilmvsm1994)"}\NormalTok{,}
  \AttributeTok{ylabel =} \StringTok{"Density"}
\NormalTok{)}
\end{Highlighting}
\end{Shaded}

\includegraphics{Taller_5_github_V1_files/figure-latex/unnamed-chunk-12-1.pdf}

\begin{Shaded}
\begin{Highlighting}[]
\FunctionTok{dev.copy}\NormalTok{(png, }\StringTok{"figura\_2b\_mccrary\_with\_hist.png"}\NormalTok{, }\AttributeTok{width =} \DecValTok{2400}\NormalTok{, }\AttributeTok{height =} \DecValTok{1800}\NormalTok{, }\AttributeTok{res =} \DecValTok{300}\NormalTok{)}
\end{Highlighting}
\end{Shaded}

\begin{verbatim}
## quartz_off_screen 
##                 3
\end{verbatim}

\begin{Shaded}
\begin{Highlighting}[]
\FunctionTok{dev.off}\NormalTok{()}
\end{Highlighting}
\end{Shaded}

\begin{verbatim}
## pdf 
##   2
\end{verbatim}

\begin{Shaded}
\begin{Highlighting}[]
\FunctionTok{cat}\NormalTok{(}\StringTok{"}\SpecialCharTok{\textbackslash{}n}\StringTok{✓ Figura con histograma guardada: figura\_2b\_mccrary\_with\_hist.png}\SpecialCharTok{\textbackslash{}n}\StringTok{"}\NormalTok{)}
\end{Highlighting}
\end{Shaded}

\begin{verbatim}
## 
## ✓ Figura con histograma guardada: figura_2b_mccrary_with_hist.png
\end{verbatim}

\textbf{Respuesta:} A partir de los resultados observados en el figura 1
(Histograma) y la figura 2 (Densidad de Kernel), no se observa un salto
en la distribución la variable \(x_i\) (ilmvsm1994) - es el margen de
votos con el que ganó o perdió el candidato del partido islámico-.
Información que concuerda con el Autor Meyersson (2014), quien no
encuentra evidencia de manipulación en la variable de asignación
alrededor del umbral (\(c=0\)).

\begin{enumerate}
\def\labelenumi{\alph{enumi})}
\setcounter{enumi}{1}
\tightlist
\item
  En una tabla presenten los resultados de estimar la ecuación de
  interés, sin controles, pero incluyendo \(f(x_i)\), tomando como
  variables dependientes: i) la elección de un alcalde del partido
  Islámico en 1984 \((i89)\) y ii) el logaritmo de la población en 1994
  \((lpop1994)\). Para esto, usen únicamente la muestra de elecciones
  alrededor del ancho de banda óptimo. Dados sus resultados, ¿parece
  haber continuidad en estas variables?
\end{enumerate}

\begin{Shaded}
\begin{Highlighting}[]
\CommentTok{\# Test 1: Islamic mayor 1989}
\NormalTok{test\_i89 }\OtherTok{\textless{}{-}} \FunctionTok{feols}\NormalTok{(}
\NormalTok{  i89 }\SpecialCharTok{\textasciitilde{}}\NormalTok{ islamic\_mayor\_1994 }\SpecialCharTok{+}\NormalTok{ ilmvsm1994 }\SpecialCharTok{+}\NormalTok{ ilmvsm1994}\SpecialCharTok{:}\NormalTok{islamic\_mayor\_1994,}
  \AttributeTok{data =}\NormalTok{ BD\_h,}
  \AttributeTok{cluster =} \SpecialCharTok{\textasciitilde{}}\NormalTok{prov}
\NormalTok{)}

\CommentTok{\# Test 2: Log población 1994}
\NormalTok{test\_lpop }\OtherTok{\textless{}{-}} \FunctionTok{feols}\NormalTok{(}
\NormalTok{  lpop1994 }\SpecialCharTok{\textasciitilde{}}\NormalTok{ islamic\_mayor\_1994 }\SpecialCharTok{+}\NormalTok{ ilmvsm1994 }\SpecialCharTok{+}\NormalTok{ ilmvsm1994}\SpecialCharTok{:}\NormalTok{islamic\_mayor\_1994,}
  \AttributeTok{data =}\NormalTok{ BD\_h,}
  \AttributeTok{cluster =} \SpecialCharTok{\textasciitilde{}}\NormalTok{prov}
\NormalTok{)}

\CommentTok{\# Tabla de tests}
\FunctionTok{cat}\NormalTok{(}\StringTok{"}\SpecialCharTok{\textbackslash{}n}\StringTok{=== TABLA 3: TESTS DE BALANCE ===}\SpecialCharTok{\textbackslash{}n}\StringTok{"}\NormalTok{)}
\end{Highlighting}
\end{Shaded}

\begin{verbatim}
## 
## === TABLA 3: TESTS DE BALANCE ===
\end{verbatim}

\begin{Shaded}
\begin{Highlighting}[]
\NormalTok{tests\_balance }\OtherTok{\textless{}{-}} \FunctionTok{list}\NormalTok{(}
  \StringTok{"Islamic Mayor 1989"} \OtherTok{=}\NormalTok{ test\_i89,}
  \StringTok{"Log Población 1994"} \OtherTok{=}\NormalTok{ test\_lpop}
\NormalTok{)}

\FunctionTok{modelsummary}\NormalTok{(}
\NormalTok{  tests\_balance,}
  \AttributeTok{stars =} \FunctionTok{c}\NormalTok{(}\StringTok{\textquotesingle{}*\textquotesingle{}} \OtherTok{=} \FloatTok{0.1}\NormalTok{, }\StringTok{\textquotesingle{}**\textquotesingle{}} \OtherTok{=} \FloatTok{0.05}\NormalTok{, }\StringTok{\textquotesingle{}***\textquotesingle{}} \OtherTok{=} \FloatTok{0.01}\NormalTok{),}
  \AttributeTok{coef\_rename =} \FunctionTok{c}\NormalTok{(}\StringTok{"islamic\_mayor\_1994"} \OtherTok{=} \StringTok{"Islamic Mayor 1994"}\NormalTok{),}
  \AttributeTok{gof\_map =} \FunctionTok{c}\NormalTok{(}\StringTok{"nobs"}\NormalTok{, }\StringTok{"r.squared"}\NormalTok{),}
  \AttributeTok{title =} \StringTok{"Tests de Continuidad en Covariables Pre{-}Tratamiento"}
\NormalTok{)}
\end{Highlighting}
\end{Shaded}

\begin{table}
\centering
\begin{talltblr}[         %% tabularray outer open
caption={Tests de Continuidad en Covariables Pre-Tratamiento},
note{}={* p \num{< 0.1}, ** p \num{< 0.05}, *** p \num{< 0.01}},
]                     %% tabularray outer close
{                     %% tabularray inner open
colspec={Q[]Q[]Q[]},
column{1}={halign=l,},
column{2}={halign=c,},
column{3}={halign=c,},
hline{10}={1,2,3}{solid, 0.05em, black},
}                     %% tabularray inner close
\toprule
& Islamic Mayor 1989 & Log Población 1994 \\ \midrule %% TinyTableHeader
(Intercept)                   & \num{0.085}*** & \num{8.087}*** \\
& (\num{0.018})  & (\num{0.139})  \\
Islamic Mayor 1994            & \num{0.003}    & \num{0.036}    \\
& (\num{0.047})  & (\num{0.156})  \\
ilmvsm1994                    & \num{0.372}*** & \num{-0.917}   \\
& (\num{0.108})  & (\num{0.907})  \\
Islamic Mayor 1994:ilmvsm1994 & \num{0.119}    & \num{2.753}    \\
& (\num{0.425})  & (\num{1.953})  \\
Num.Obs.                      & \num{719}      & \num{970}      \\
R2                            & \num{0.038}    & \num{0.003}    \\
\bottomrule
\end{talltblr}
\end{table}

\begin{Shaded}
\begin{Highlighting}[]
\CommentTok{\# Extraer p{-}values}
\NormalTok{pval\_i89 }\OtherTok{\textless{}{-}} \FunctionTok{summary}\NormalTok{(test\_i89)}\SpecialCharTok{$}\NormalTok{coeftable[}\StringTok{"islamic\_mayor\_1994"}\NormalTok{, }\StringTok{"Pr(\textgreater{}|t|)"}\NormalTok{]}
\NormalTok{pval\_lpop }\OtherTok{\textless{}{-}} \FunctionTok{summary}\NormalTok{(test\_lpop)}\SpecialCharTok{$}\NormalTok{coeftable[}\StringTok{"islamic\_mayor\_1994"}\NormalTok{, }\StringTok{"Pr(\textgreater{}|t|)"}\NormalTok{]}

\FunctionTok{cat}\NormalTok{(}\StringTok{"}\SpecialCharTok{\textbackslash{}n}\StringTok{=== RESULTADOS TESTS ===}\SpecialCharTok{\textbackslash{}n}\StringTok{"}\NormalTok{)}
\end{Highlighting}
\end{Shaded}

\begin{verbatim}
## 
## === RESULTADOS TESTS ===
\end{verbatim}

\begin{Shaded}
\begin{Highlighting}[]
\FunctionTok{cat}\NormalTok{(}\StringTok{"Islamic Mayor 1989:}\SpecialCharTok{\textbackslash{}n}\StringTok{"}\NormalTok{)}
\end{Highlighting}
\end{Shaded}

\begin{verbatim}
## Islamic Mayor 1989:
\end{verbatim}

\begin{Shaded}
\begin{Highlighting}[]
\FunctionTok{cat}\NormalTok{(}\StringTok{"  Coeficiente:"}\NormalTok{, }\FunctionTok{round}\NormalTok{(}\FunctionTok{coef}\NormalTok{(test\_i89)[}\StringTok{"islamic\_mayor\_1994"}\NormalTok{], }\DecValTok{4}\NormalTok{), }\StringTok{"}\SpecialCharTok{\textbackslash{}n}\StringTok{"}\NormalTok{)}
\end{Highlighting}
\end{Shaded}

\begin{verbatim}
##   Coeficiente: 0.0033
\end{verbatim}

\begin{Shaded}
\begin{Highlighting}[]
\FunctionTok{cat}\NormalTok{(}\StringTok{"  P{-}value:"}\NormalTok{, }\FunctionTok{round}\NormalTok{(pval\_i89, }\DecValTok{4}\NormalTok{), }\StringTok{"}\SpecialCharTok{\textbackslash{}n}\StringTok{"}\NormalTok{)}
\end{Highlighting}
\end{Shaded}

\begin{verbatim}
##   P-value: 0.945
\end{verbatim}

\begin{Shaded}
\begin{Highlighting}[]
\FunctionTok{cat}\NormalTok{(}\StringTok{"}\SpecialCharTok{\textbackslash{}n}\StringTok{Log Población 1994:}\SpecialCharTok{\textbackslash{}n}\StringTok{"}\NormalTok{)}
\end{Highlighting}
\end{Shaded}

\begin{verbatim}
## 
## Log Población 1994:
\end{verbatim}

\begin{Shaded}
\begin{Highlighting}[]
\FunctionTok{cat}\NormalTok{(}\StringTok{"  Coeficiente:"}\NormalTok{, }\FunctionTok{round}\NormalTok{(}\FunctionTok{coef}\NormalTok{(test\_lpop)[}\StringTok{"islamic\_mayor\_1994"}\NormalTok{], }\DecValTok{4}\NormalTok{), }\StringTok{"}\SpecialCharTok{\textbackslash{}n}\StringTok{"}\NormalTok{)}
\end{Highlighting}
\end{Shaded}

\begin{verbatim}
##   Coeficiente: 0.0363
\end{verbatim}

\begin{Shaded}
\begin{Highlighting}[]
\FunctionTok{cat}\NormalTok{(}\StringTok{"  P{-}value:"}\NormalTok{, }\FunctionTok{round}\NormalTok{(pval\_lpop, }\DecValTok{4}\NormalTok{), }\StringTok{"}\SpecialCharTok{\textbackslash{}n}\StringTok{"}\NormalTok{)}
\end{Highlighting}
\end{Shaded}

\begin{verbatim}
##   P-value: 0.8173
\end{verbatim}

\textbf{Respuesta:} A partir de los resultados presentados en la tabla 3
(Tests de Balance), se realiza un prueba de hipótesis para cada una de
las variables dependientes:

Ho: No hay discontinuidad con alcalde islámico 5 años atras. El p-valor
es mayor a 0.10, por lo tanto, no se rechaza Ho. Los municipios que
estan cerca al umbral son similares en terminos de la variable i89.

Ho: No hay discontinuidad en tamaño poblacional. El p-valor es mayor a
0.10, por lo tanto, no se rechaza Ho. Los municipios pequeños o grandes
tienen igual probabilidad de estar cerca del umbral.

En conclusión, no hay evidencia de discontinuidad en las variables
analizadas en el tabla 3, toda vez las pruebas de hipotesis resultaron
en no rechazo de la Ho.

\begin{enumerate}
\def\labelenumi{\alph{enumi})}
\setcounter{enumi}{2}
\tightlist
\item
  Dados sus resultados en los incisos a y b, ¿es plausible el supuesto
  de identificación? Expliquen por qué.
\end{enumerate}

\textbf{Respuesta:} El supuesto de identificación para una RD, se basa
en la manipulación y el test de balance. Luego de corroborar estas
mediciones que se resolvieron en los puntos a y b, se puede concluir que
el supuesto de identificación es plausible, ya que:

\begin{enumerate}
\def\labelenumi{\arabic{enumi}.}
\item
  No existe evidencia de manipulación, la distribución es continua en el
  umbral (\(c=0\)). Esto sugiere que los candidatos no pudieron
  controlar los resultados electorales.
\item
  Balanceo, la caracteristicas previas al tratamiento (elección de
  alcalde islámico en 1989 y tamaño poblacional en 1994) no presentan
  discontinuidad en el umbral (\(c=0\)). Esto indica que los municipios
  cercanos al umbral son comparables.
\end{enumerate}

\end{document}
